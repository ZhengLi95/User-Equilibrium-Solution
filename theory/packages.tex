\usepackage{lmodern}
\usepackage[utf8]{inputenc}
\usepackage[T1]{fontenc}

%%%%%%%%%%%%%%%%%%%% T E X T E %%%%%%%%%%%%%%%%%%%%%%

% package qui permettent d'utiliser tout un tas de symboles mathématiques
\usepackage{amsmath}
\numberwithin{equation}{section}

\usepackage{amssymb}
\usepackage{empheq}
\usepackage{physics}
\usepackage[squaren,Gray]{SIunits}

%ceiling function
\usepackage{mathtools}
\DeclarePairedDelimiter{\ceil}{\lceil}{\rceil}

\usepackage{csquotes}

% Polices
\usepackage{mathpazo}

% package qui permettent de personaliser les puces des listes
\usepackage{enumerate}
\usepackage{enumitem}

% package qui permettent d'introduire de la couleur
\usepackage{color}
\definecolor{mygreen}{rgb}{0,0.6,0}
\definecolor{mygray}{rgb}{0.5,0.5,0.5}
\definecolor{mymauve}{rgb}{0.58,0,0.82}

\usepackage{xspace}

% pour pouvoir barrer avec \cancel{}
\usepackage{cancel}

% pour la fonction indicatrice 1
\usepackage{dsfont}

% pour utiliser le symbole euro par \euro
\usepackage{eurosym}

% afficher des nombres en mode texte
\usepackage{numprint}

% to insert notes in the margin. Can be useful during the editorial process
\usepackage{marginnote}
\usepackage[top=1.5cm, bottom=1.5cm, outer=5cm, inner=2cm, heightrounded, marginparwidth=2.5cm, marginparsep=1cm]{geometry}


%%%%%%%%%%%%%%%%%%%% F I G U R E S %%%%%%%%%%%%%%%%%%%%%%

% package qui permettent d'inclure des graphiques
%\usepackage{graphicx}
%\numberwithin{figure}{section}

%Pour utiliser des sous-figures
%\usepackage{subcaption}

%Pour utiliser [H] pour le placement des figures
%\usepackage{float}

%figure entourée de texte
%\usepackage{wrapfig}
% \begin{wrapfigure}{L}{0.4\textwidth}

%Keep figures in section
%\usepackage[section]{placeins}
% \FloatBarrier % stops floats from descending further

% Pour des références stylées
%\usepackage{varioref}
%\usepackage[german]{fancyref} % dispo seulement en allemand
\usepackage[linkcolor=black,
						urlcolor=black,
						colorlinks=true,
						citecolor=mygreen]{hyperref}% package qui créer des liens dynamiques dans un fichier pdf à partir de tout élément taggé
%\usepackage{cleveref}


%%%%%%%%%%%%%%%%%%%% D E S S I N %%%%%%%%%%%%%%%%%%%%%%

%pour dessiner des graphes de fonctions
%\usepackage{pgfplots}

%%%%%%%%%%%%%%%%%%%% H E A D E R et F O O T E R %%%%%%%%%%%%%%%%%%%%%%

% \usepackage{fancyhdr}
% \fancyfoot{}
% \fancyfoot[LE,RO]{\thepage}


%%%%%%%%%%%%%%%%%%%%%% D I V E R S %%%%%%%%%%%%%%%%%%%%%%%%%%

%Bibliographie
\usepackage[backend=bibtex,sorting=ynt]{biblatex}%sorting=none pour avoir les références selon d'ordre d'apparition dans le texte
% sorting=ynt		backend=biber,		citestyle=alphabetic		url=false

% Set a watermark
% \usepackage{draftwatermark}
% \SetWatermarkText{© Till \textsc{Kletti} et al.}
% \SetWatermarkFontSize{1cm}
% \SetWatermarkScale{1}
% \SetWatermarkLightness{1} % white for invisibility
% \SetWatermarkAngle{0}

%%%%%%%%%%%%%%%%%%% C O D E %%%%%%%%%%%%%%%%%%%%%%%

% Pour écrire du pseudo-code
\usepackage{algpseudocode}
\usepackage{algorithm}
\algdef{SE}[DOWHILE]{Do}{doWhile}{\algorithmicdo}[1]{\algorithmicwhile\ #1}

% pour insérer du code
\usepackage{listings}
\usepackage{listingsutf8}